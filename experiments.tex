\documentclass[12pt, notitlepage]{article}
\usepackage{url}
\usepackage{amssymb}
\begin{document}
\bibliographystyle{plain}
\title{Experimenting on HTTP/2}
\maketitle
\begin{abstract}
This short paper aims at comparing the performances of HTTP/1.1 and HTTP/2.
The metric used to evaluate the performances is the Page Loading Time
(PLT) as defined in ~\cite{w3c}.
To sum it up, the PLT is the time elapsed between the moment the user
enters a URL in her search bar and the moment the webpage requested is
fully displayed in the browser window. We also try to understand how the
network characteristics influence the page loading times on these two
versions of HTTP (mobile networks in particular).
\end{abstract}

\section{Introduction}

% TODO: Introduce what is HTTP, its limits and the reason why HTTP/2
%		was introduced
% Only support of Ubuntu Trusty in the experiments
We analyze two versions of the HTTP protocol: HTTP/1.1 and HTTP/2.
For both protocols, we study two variants: a cleartext one and an
encrypted other. We end up with four protocols summarized as follows:
\begin{itemize}
	\item[--] HTTP/1.1 cleartext $\to$ HTTP;
	\item[--] HTTP/1.1 secure $\to$ HTTPS;
	\item[--] HTTP/2 cleartext $\to$ h2c;
	\item[--] HTTP/2 secure $\to$ h2.
\end{itemize}
\newpage
\section{Setting up the experiments}

To compare the page loading times we need to set up: 
\begin{itemize}
\item[--] a server, to provide the webpages;
\item[--] a client, to request and load them.
\end{itemize}

The client and the server used throughout our experiments need to be the
same for all protocols -- else our results would be biased. How should we
choose them?
The page loading times will depend on our choice of client and server. But
since we are only considering the difference of PLTs between protocols, it
does not matter here.

\subsection{Set up a server}

The problem is that there is no server delivering real content on all four
protocols.
% not even nghttp2.org (nghttp2 by Tatsuhiro) because he is using a proxy
% maybe check http2rulez.com --> they might have a service running on all
% four of them protocols.
That is why we need to setup a server which will serve content on the 
four protocols. But we still have to choose which content we want to 
deliver. We decided to the most consulted websites on Alexa~\cite{alexa} 
and cloned the main pages (Google, Facebook, Youtube, etc) on our own 
machine.
% using ScrapBook
% Some content that were cloned do not reflect real loading of webpages:
% for instance loading the Facebook login page has nothing to do with
% loading a Facebook wall.

Also on this machine we have to set up a server able to provide these
contents on all four protocols. Looking at the list of available 
implementations at~\cite{implem}, only one server fits these requirements:
H2O~\cite{h2o}.\\

% TODO: Create an Appendix explaining how to set up H2O on ubuntu Trusty
If you wish to understand better how the setting up of H2O works, you
can look at~\cite{h2o}. However, we already did all the job for you: a
server is up and running at the IP address later, waiting for you to
retrieve its content. And the only thing you need to do that is a client.

\subsection{Set up a client}

The client is just a browser able to talk all four of the above 
protocols. Currently, no browser provides HTTP/2 cleartext support; so we
picked Chromium (which already talked HTTP, HTTPS and h2) and
slightly modified it so that it could talk h2c (h2 without the TLS 
layer).\\

Here again, no need to build Chromium yourself, because we have the
modified binaries available at ~\cite{chromium}. Feel free to download and
install this package on your latest Ubuntu version (for any other OS, we 
do not have the executables yet).

\section{Tool for experimenting}

The tool to test the page load time performances is available at
~\cite{load_times}. All you need to do is to clone this repository and
to use the utility in it. For instance you can try: 
./test --ip later -s 30 -t 1 -r leopard.html. Be careful, if you are
behind a proxy some protocols might not be authorized (typically h2c).
\bibliography{references}
\end{document}


