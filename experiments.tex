\documentclass[12pt, notitlepage]{article}

\usepackage{url}		% bibliography: inserting an url
\usepackage{amssymb}	% usual
\usepackage{listings}	% displaying bash command lines
\usepackage{enumitem}	% no separation between items in itemize
						% need to install texlive-extra
\begin{document}
\bibliographystyle{plain}

\title{Experimenting on HTTP/2}
\maketitle

\begin{abstract}
This short paper aims at comparing the performances of HTTP/1.1 and HTTP/2.
The metric used to evaluate the performances is the Page Loading Time
(PLT) as defined in ~\cite{w3c}.
To sum it up, the PLT is the time elapsed between the moment the user
enters a URL in her search bar and the moment the webpage requested is
fully displayed in the browser window. We also try to understand how the
network characteristics influence the page loading times on these two
versions of HTTP (mobile networks in particular).
\end{abstract}

\section{Introduction}

% TODO: Introduce what is HTTP, its limits and the reason why HTTP/2
%		was introduced
% Only support of Ubuntu Trusty in the experiments
We analyze two versions of the HTTP protocol: HTTP/1.1 and HTTP/2.
For both protocols, we study two variants: a cleartext one and an
encrypted other. We end up with four protocols summarized as follows:
\begin{itemize}[noitemsep]
	\item[--] HTTP/1.1 cleartext $\to$ HTTP;
	\item[--] HTTP/1.1 secure $\to$ HTTPS;
	\item[--] HTTP/2 cleartext $\to$ h2c;
	\item[--] HTTP/2 secure $\to$ h2.
\end{itemize}
\newpage
\section{Setting up the experiments}

To compare the page loading times we need to set up: 
\begin{itemize}[noitemsep]
\item[--] a server, to provide the webpages;
\item[--] a client, to request and load them.
\end{itemize}

The client and the server used throughout our experiments need to be the
same for all protocols -- else our results would be biased. How should we
choose them?
% The page loading times will depend on our choice of client and server. But
% since we are only considering the difference of PLTs between protocols, it
% does not matter here.

\subsection{Setting up a server}

Why do we need to set up a server when there are thousands of webservers
already delivering contents on the Internet? Well, the problem is that
those servers only use three of the aforementioned protocols: HTTP, HTTPS
and h2. But h2c is never used.
% not even nghttp2.org (nghttp2 by Tatsuhiro) because he is using a proxy
% maybe check http2rulez.com --> they might have a service running on all
% four of them protocols.
That is why we need to setup a server that will serve content on the 
four protocols. 
% using ScrapBook
% Some content that were cloned do not reflect real loading of webpages:
% for instance loading the Facebook login page has nothing to do with
% loading a Facebook wall.
Looking at the list of available 
implementations on~\cite{implem}, only one server fits these requirements:
H2O~\cite{h2o}.\\

% TODO: Create an Appendix explaining how to set up H2O on ubuntu Trusty
If you wish to better understand how the setting up of H2O works, you
can look at~\cite{h2o}. However, we already did all the job for you: a
server is up and running at the IP address 161.106.2.57.
But we still have to choose which content we want to 
deliver: we decided to take the most consulted websites on
Alexa~\cite{alexa} and to clone the main pages (Google, Facebook, Youtube,
etc) on our own machine.
Right now, this server is just waiting for you to
retrieve its content. And the only thing you need to do that is a client.

\subsection{Setting up a client}

The client is just a browser able to talk all four of the above 
protocols. Currently, no browser provides h2c (HTTP/2 cleartext) support;
so we picked Chromium (which already talked HTTP, HTTPS and h2) and
slightly modified it so that it could talk h2c.\\

\newpage
Here again, no need to build Chromium yourself, because we have the
modified binaries available at~\cite{chromium}. Feel free to download and
install this package on your latest Ubuntu version (no other OS support at 
the moment). Here is how you can make it work:
\begin{lstlisting}[language=bash]
$ cd my/favorite/workspace 
$ git clone https://github.com/desaxce/chromium 
$ cd chromium 
$ tar -jxvf chromium.tar.bz2 
$ cd chromium-browser 
$ chmod a+x chromium-browser nacl_helper nacl_irt_x86_64.nexe 
$ sudo chown root:root chrome_sandbox 
$ sudo chmod 4755 chrome_sandbox 
$ export CHROME__DEVEL__SANDBOX="$PWD/chrome_sandbox"
\end{lstlisting}

Then you can check if Chromium works by doing: 
\begin{lstlisting}[language=bash]
$ ./chromium-browser
\end{lstlisting}

\section{Tool for experimenting}
% Installation for Ubuntu Trusty Tahr only
The tool to test the page load time performances is available at
~\cite{load_times}. Go to the same workplace where you cloned chromium
and clone this other repository with the following command line: 
\begin{lstlisting}[language=bash]
  $ git clone https://github.com/desaxce/load_times
  $ sudo apt-get install build-essential gnuplot gnuplot-x11
\end{lstlisting}
\vspace*{0.5cm}

Now you are all set. You can try: 
\begin{lstlisting}[language=bash]
$ make
$ ./test --ip 161.106.2.57 -s 30 -t 1 -r leopard.html
\end{lstlisting}

\vspace*{0.5cm}
This will: 
\begin{itemize}[noitemsep]
\item[--] retrieve the webpage leopard.html at the address 
161.106.2.57;
\item[--] wait exactly 30 seconds for the page to load;
\item[--] request the webpage exactly one time for each\footnote{If you are
behind a proxy some protocols might be filtered (typically h2c)}
protocol.
\end{itemize}

The results are written to 0.0.default.161.106.2.57.txt ("0.0" stands for
0\% delay and 0\% packet loss: those are parameters you can change). To
compare the page loading times, you simply have to execute:
\begin{lstlisting}[language=bash]
$ ./hist_total.gnu 0.0.default.161.106.2.57.txt
\end{lstlisting}
For more information about this small utility type:
\begin{lstlisting}[language=bash]
$ ./test
\end{lstlisting}
\newpage
\bibliography{references}
\end{document}


